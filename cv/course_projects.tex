%!TEX root = ../cv.tex
\cvsection{Course Projects}

\begin{cventries}
	% \cventry
	% 	{Course: Computational Structural Dynamics, Instructor: Prof. Salil Kulkarni, Department of Mechanical Engineering}
	% 	{Analytical selection of masters for the reduced eigenvalue problem} 
	% 	{}
	% 	{January 2017 - April 2017}
	% 	{
	% 	\begin{cvitems}
	% 		\item Studied algorithm of selection of master nodes for Guyan Reduction and implemented two algorithms on a $10^5 \times 10^5$ matrix in GNU Octave
	% 	\end{cvitems}
	% 	}

	\cventry
		{Course: Adaptive Control Theory, Instructor: Prof. Sukumar Srikant, Systems and Controls Engineering}
		{Adaptive Control under Input Constraints}
		{}
		{January 2017 - April 2017}
		{
		\begin{cvitems}
			\item Studied paper (Positive $\mu$- modification for stable adaptation in the presence of input constraints) of $\mu$-modification proposed by Eugene Lavretsky and Naira Hovakimyan for Input constraints
			\item Studied paper (Adaptive tracking for nonlinear systems with control constraints) by Alexander Leonessa, Wassim M Haddad, and Tomohisa Hayakawa for input saturation and input rate saturation constraints
			\item Performed simulation of both the papers on a system in GNU Octave
			\item Compared results between both of them listing pros and cons of the strategies 
		\end{cvitems}
		}

	\cventry
		{Course: Acoustics and Hearing, Instructor: Prof. Sripriya Ramamoorthy, Department of Mechanical Engineering}
		{Active Noise Cancellation - Coding LMS and Modified LMS Methods}
		{}
		{January 2017 - April 2017}
		{
		\begin{cvitems}
			\item Studied various types of noise cancellation algorithms, specifically Least Mean Square algorithm used for active noise cancellation using two microphones. LMS and Modified LMS methods were implemented in GNU Octave for performing simulations
			\item Voice samples of students mess, engine sounds, songs, white noise(computer generated), and recorded voice were used to look at the attenuation. Attenuation of upto \SI{19}{\decibel} was observed in simulation of random noise. The results of both the algorithms were compared
		\end{cvitems}
		}

	% \cventry
	% 	{Course: Computer Aided Simulation of Machines, Instructor: Prof. Anirban Guha, Department of Mechanical Engineering}
	% 	{Kinematic and Dynamic study of Geneva Mechanism}
	% 	{}
	% 	{January 2017 - April 2017}
	% 	{
	% 	\begin{cvitems}
	% 		\item Studied design of Geneva Mechanism and performed dynamic and kinematic simulation in MSC Adams; Verified results theoretically
	% 	\end{cvitems}
	% 	}

	\cventry
		{Course: Geometric and Analytic Aspects of Optimal Control, Instructor: Prof. Ravi Banavar, Systems and Controls Engineering}
		{Paper Review, Optimal Control}
		{}
		{January 2017 - April 2017}
		{
		\begin{cvitems}
			\item Studied paper (A Simple Proof of the Pontryagin maximum principle) by Dong Eui Chang and prepared a report of the same
			\item Studied paper (Time-optimal control of a 3-level quantum system and its generalization) by Dong Eui Chang and presented it.
		\end{cvitems}
		}

	\cventry
		{Course: Machine Design, Instructor: Prof. Shantanu Tripathi, Department of Mechanical Engineering}
		{Passive Walking Assist Device}
		{}
		{July 2017 - November 2017}
		{
		\begin{cvitems}
			\item Designed a device which would assist people with weaker thigh muscles or injury to walk without any external energy source
			\item Studied energy required in different parts of the human gait using openly available medical data and simulated it in MSC ADAMS
			\item Designed a catch and release mechanism to release spring when the foot would do negative work to minimise strain on user
			\item Fabricated the design made in solidworks of the said mechanism using laser cutting and made a working prototype
		\end{cvitems}
		}

	\cventry
		{Course: Design of Mechatronic Systems, Instructor: Prof. Prasanna Gandhi, Department of Mechanical Engineering}
		{Design of a Spherical Robot}
		{}
		{July 2017 - November 2017}
		{
		\begin{cvitems}
			\item Designed in a team of two a spherical robot in SolidWorks with rotation and stability based on gyroscopic effect and calculated specifications required for various actuators used in the robot for specified values of velocity
			\item Fabricated the bot interior using Laser Printing and achieved desired speed of the bot by open loop control controlling it through bluetooth by developing a bluetooth control system using motor drivers, ATmega32 microcontroller chip and bluetooth module
		\end{cvitems}
		}

	% \cventry
	%:::%:::,,::,,::,,:	}, 	{Minesweeper,	%, 	{: 	,Course:{},:Rapid Product Development, Instructor: Prof. K.P. Karunakaran, Department of Mechanical Engineering}
	%	{}
	% 	{July 2017 - November 2017}
	% 	{
	% 	\begin{cvitems}
	% 		\item Designed and Programmed the popular game Minesweeper in C++ using Allegro Graphics Library.
	% 	\end{cvitems}
	% 	}

	% \cventry
	%:::%:::,,::,,::,,:	}, 	{Minesweeper,	%, 	{: 	,Course:{},:Computer Graphics and Product Modelling, Instructor: Prof. S.S. Pande, Department of Mechanical Engineering}
	%	{}
	% 	{July 2017 - November 2017}
	% 	{
	% 	\begin{cvitems}
	% 		\item Designed and Programmed the poopular game Minesweeper in C++ using Allegro Graphics Library.
	% 	\end{cvitems}
	% 	}

	% \cventry
	% 	{Course: Design Optimization, Instructor: Prof. Salil Kulkarni, Department of Mechanical Engineering}
	% 	{Minimisation of Weight of Quadcopter for fixed payload}
	% 	{}
	% 	{July 2017 - November 2017}
	% 	{
	% 	\begin{cvitems}
	% 		\item Minimised the design parameters of a quadcopter optimising its weight for carrying a payload of 1 kg; Performed multiple iterations for solving the optimal value problem studying the effect of the value of the initial guess
	% 	\end{cvitems}
	% 	}

	\cventry
		{Course: Manufacturing Processes, Instructor: Prof. Ramesh Singh, Department of Mechanical Engineering}
		{FEA of Drilling of CFRP Polymers}
		{Studied drilling of orthotropic composite fibre reinforced polymers and performed FEA simulation in ABAQUS. Matched value of delamination with that available in literature at various feed rates and thrust forces}
		{July 2016 - November 2016}
		{
		% \begin{cvitems}
		% 	\item Studied drilling of orthotropic composite fibre reinforced polymers sheets to understand delamination of said material due to drilling 
		% 	\item CFRP was modelled using Johnson Cook model and drill bit model was developed in SolidWorks. FEA simulation was performed in ABAQUS.
		% 	\item Delamination was found at varying feed rates and drilling thrust forces and the equations found during literature survey were verified
		% \end{cvitems}
		}

	% \cventry
	%:::%:::,,::,,::,,:	}, 	{Minesweeper,	%, 	{: 	,Course:{},:Computer Programming and Utilization, Instructor: Prof. D.B. Phatak}
	%	{}
	% 	{July 2014 - November 2014}
	% 	{
	% 	\begin{cvitems}
	% 		\item Designed and Programmed the popular game Minesweeper in C++ using Allegro Graphics Library.
	% 	\end{cvitems}
	% 	}
\end{cventries}